\documentclass[12pt,a4paper]{article}
\usepackage[utf8]{inputenc}
\usepackage[T2A]{fontenc}
\usepackage[english,russian]{babel}
\usepackage[a4paper, mag=1000, left=1.5cm, right=2cm, top=2cm, bottom=2cm, headsep=0.7cm, footskip=1cm]{geometry}
\usepackage{amsmath}
\usepackage{pgfplots, colortbl}
\usepackage{makecell}
\usepackage{multicol}
\usepackage{pgfplotstable}
\pgfplotsset{compat=1.16}
\usepackage{minted}
\usepackage{listings}
\usepackage{lstfiracode}
\usepackage{caption}
\usepackage{mathrsfs}
\usepackage{placeins}
\usepackage{graphicx}

\usemintedstyle{colorful}
\newenvironment{code}{\captionsetup{type=listing}}{}


\newcommand{\itertable}[2]{
	\FloatBarrier
	\begin{table}[h]
		\centering
		\caption*{Количество точек}
		\pgfplotstabletypeset[
			every even row/.style=
				{before row={\rowcolor[gray]{0.95}}},
			string type,
			columns/point/.style={column name=Начальная точка, column type={|c|}},
			columns/f1/.style={column name=$f_1$, column type={c|}},
			columns/f2/.style={column name=$f_2$, column type={c|}},
			columns/f3/.style={column name=$f_3$, column type={c|}},
			columns/f4/.style={column name=$f_4$, column type={c|}},
			every head row/.style={before row=\hline, after row=\hline},
			every last row/.style={after row=\hline}
		]{#1}
	\end{table}
	\FloatBarrier
}

\newcommand{\dtable}[2]{
	\FloatBarrier
	\begin{table}[h]
		\centering
		\caption*{#2}
		\pgfplotstabletypeset[
			every even row/.style=
			{before row={\rowcolor[gray]{0.95}}},
			string type,
			columns/iter/.style={column name=$№$, column type={|c|}},
			columns/iter/.style={column name=$\alpha$, column type={c|}},
			every head row/.style={before row=\hline, after row=\hline},
			every last row/.style={after row=\hline}
		]{#1}
	\end{table}
	\FloatBarrier
}

\newcommand{\methodtable}[2]{
	\FloatBarrier
	\begin{table}[h]
		\centering
		\caption*{Количество точек}
		\pgfplotstabletypeset[
			every even row/.style=
			{before row={\rowcolor[gray]{0.95}}},
			string type,
			columns/method/.style={column name=Метод, column type={|c|}},
			columns/f1/.style={column name=$f_1$, column type={c|}},
			columns/f2/.style={column name=$f_2$, column type={c|}},
			every head row/.style={before row=\hline, after row=\hline},
			every last row/.style={after row=\hline}
		]{#1}
	\end{table}
	\FloatBarrier
}



\newcommand{\loggraph}[3]{
	\begin{center}
		\begin{tikzpicture}
			\begin{semilogxaxis}[
					title = {График зависимости количества итераций метода от размерности},
					xlabel = $\log n$,
					ylabel = \(\log iterations\),
					ylabel style={rotate=-90},
					ymode = log,
					legend pos=outer north east
				]
				\addplot table [x={n}, y={iter}, /pgf/number format/read comma as period] {#1};
				\addplot table [x={n}, y={iter}, /pgf/number format/read comma as period] {#2};
				\addplot table [x={n}, y={iter}, /pgf/number format/read comma as period] {#3};
				\addlegendentry{диагональное преобладание}
				\addlegendentry{обратный знак}
				\addlegendentry{матрицы Гильберта}
			\end{semilogxaxis}
		\end{tikzpicture}
	\end{center}
}


\newcommand{\img}[1] {
	\begin{center}
		\includegraphics[width=.7\linewidth, height=.4\textheight]{#1}
	\end{center}
}
\newcommand{\mcode}[2]{
	\begin{code}
		\caption*{#1}
		\inputminted[breaklines=true, xleftmargin=1em, linenos, frame=single, framesep=10pt, fontsize=\footnotesize]{cpp}{#2}
	\end{code}
	\newpage
}


\begin{document}



\begin{titlepage}
    \begin{center}
        \textsc{Национальный исследовательский университет ИТМО\\
            Прикладная математика и информатика}\\[5cm]

        \huge{Методы оптимизации\\[6mm]
            \large Отчет по лабораторной работе №1\\
            ``Алгоритмы одномерной минимизации функции''\\[6mm]
            Вариант 1
            \\[3cm]
        }
    \end{center}

    \begin{flushright}
        \begin{minipage}{0.25\textwidth}
            Выполнили:\\[2mm]
            Михайлов Максим\\
            Загребина Мария\\
            Кулагин Ярослав\\[2mm]
            Команда:

            \(\forall \bar R \in \mathscr{R}^n : \mathrm{\textbf{R}}(\bar R) \in \mathscr{R}\)

            (КаМаЗ)\\[2mm]
            Группа: M3237
        \end{minipage}
    \end{flushright}

    \vfill
    \begin{center}
        Санкт-Петербург, \today
    \end{center}
\end{titlepage}





\section{Цель}
Реализовать алгоритмы одномерной минимизации функции:
\begin{itemize}
    \item Метод дихотомии
    \item Метод золотого сечения
    \item Метод Фибоначчи
    \item Метод парабол
    \item Комбинированный метод Брента
\end{itemize}
Протестировать алгоритмы на \(f(x) = x^2 + e^{-0.35x}\) в интервале \([-2; 3]\) и других функциях, сравнить методы друг с другом.




\section{Ход работы}

\subsection{Аналитическое решение}

\begin{multicols}{2}
    \begin{align*}
        0 = f'(x)        & = 2x - 0.35 e^{ - 0.35x}                          \\
        2x               & = 0.35 e^{ - 0.35x}                               \\
        800 \cdot 0.35 x & = 49 \frac{1}{e^{0.35 x}}                         \\
        0.35 x : = W(z)                                                      \\
        800 \cdot W(z)   & = 49 \frac{W(z)}{z}                               \\
        \frac{49}{800}   & = z                                               \\
        0.35 x           & = W\left(\frac{800}{49}\right)                    \\
        x                & = \frac{20}{7} \cdot W\left(\frac{800}{49}\right)
    \end{align*}

    \(W\)-функция Ламберта не может быть выражена в элементарных функциях, поэтому аналитическое решение приближенное:

    \[\begin{cases}
            x_{\min} \approx 0.16517 \\
            y_{\min} \approx 0.9711  \\
        \end{cases}\]
    \columnbreak

    \begin{tikzpicture}
        \begin{axis}[
                title = \(x^2 + e^{-0.35x}\),
                xmin = -2,
                xmax = 3,
            ]
            \addplot[blue, samples=50] {x^2 + e^(-0.35*x)};
        \end{axis}
    \end{tikzpicture}
\end{multicols}



\subsection{Метод дихотомии}

\stats{data/dichotomy.csv}\\[2mm]
\graph{data/dichotomy2.csv}

\subsection{Метод золотого сечения}

\stats{data/goldenRatio.csv}\\[2mm]
\graph{data/goldenRatio2.csv}

\subsection{Метод Фибоначчи}

\stats{data/fibonacci.csv}\\[2mm]
\graph{data/fibonacci2.csv}


\subsection{Метод парабол}

\stats{data/parabola.csv}\\[2mm]
\graph{data/parabola2.csv}

\subsection{Комбинированный метод Брента}
\pgfplotstabletypeset[
    every even row/.style=
        {before row={\rowcolor[gray]{0.95}}},
    string type,
    columns/number/.style={column name=$N$, column type={|l}},
    columns/a/.style={column name=$a$, column type={|c}},
    columns/b/.style={column name=$b$, column type={|c|}},
    columns/length/.style={column name=\makecell[b]{\% длины\\предыдущего\\отрезка}, column type={c|}},
    columns/u/.style={column name=$u$, column type={c|}},
    columns/w/.style={column name=$w$, column type={c|}},
    columns/x/.style={column name=$x$, column type={c|}},
    columns/v/.style={column name=$v$, column type={c|}},
    columns/fu/.style={column name=$f(u)$, column type={c|}},
    columns/fw/.style={column name=$f(w)$, column type={c|}},
    columns/fx/.style={column name=$f(x)$, column type={c|}},
    columns/fv/.style={column name=$f(v)$, column type={c|}},
    every head row/.style={before row=\hline, after row=\hline},
    every last row/.style={after row=\hline},
]{data/brent.csv}

\graph{data/brent2.csv}

\section{Тестирование на многомодальных функциях}

\subsection{\(f(x) = \sin(x)\cdot x\)}
\begin{tikzpicture}
    \begin{axis}[
            xmin = -5,
            xmax = 5,
        ]
        \addplot[blue, samples=50] {sin(deg(x))*x};
    \end{axis}
\end{tikzpicture}

\begin{tabular}{ | l | l | l | }
    \hline
                    & \(f(x) = \sin(x)\cdot x\) & \\ \hline
    Верный ответ    & 4.91318                   & \\ \hline
    Дихотомия       & 2.2594e-16                & \\ \hline
    Золотое сечение & 2.80886e-16               & \\ \hline
    Фибоначчи       & 2.2594e-16                & \\ \hline
    Параболы        & 4.91318                   & \\ \hline
    Метод Брента    & 4.91318                   & \\ \hline
\end{tabular}

\section{Выводы}

\begin{center}
    \begin{tikzpicture}
        \begin{semilogxaxis}[
                title = {Общий график},
                xlabel = $\varepsilon$,
                ylabel = $n$,
                width=0.7\textwidth
            ]
            \addplot table [x={eps}, y={number}] {data/dichotomy2.csv};
            \addplot table [x={eps}, y={number}] {data/goldenRatio2.csv};
            \addplot table [x={eps}, y={number}] {data/fibonacci2.csv};
            \addplot table [x={eps}, y={number}] {data/parabola2.csv};
            \addplot table [x={eps}, y={number}] {data/brent2.csv};
            \addlegendentry{Дихотомия}
            \addlegendentry{Золотое сечение}
            \addlegendentry{Фибоначчи}
            \addlegendentry{Метод парабол}
            \addlegendentry{Метод Брента}
        \end{semilogxaxis}
    \end{tikzpicture}
\end{center}

Для каждого метода $\varepsilon = 10^{-6}$\\

Из рассмотренных алгоритмов быстрее всего сходится комбинированный алгоритм Брента, т.к. он сочетает в себе преимущества метода парабол и золотого сечения --- квадратичная сходимость в окрестности решения и гарантированно линейная сходимость вне окрестности.

Рассмотренные алгоритмы не способны выполнять нахождение глобального многомодальных функций в общем случае, но находят локальный.

\newpage
\section{Исходный код}

\subsection{Рассматриваемый отрезок}
\mcode{../include/lab/segment.h}
\mcode{../source/segment.cpp}

\subsection{Общий класс оптимизаторов}
\mcode{../include/lab/optimizer.h}
\mcode{../source/optimizer.cpp}
\newpage

\subsection{Общий класс оптимизаторов, рассматривающих 2 точки}
\mcode{../include/lab/two_point.h}
\mcode{../source/two_point.cpp}
\newpage

\subsection{Метод дихотомии}
\mcode{../include/lab/dichotomy.h}
\mcode{../source/dichotomy.cpp}
\newpage

\subsection{Метод золотого сечения}
\mcode{../include/lab/golden_ratio.h}
\mcode{../source/golden_ratio.cpp}
\newpage

\subsection{Метод Фибоначчи}
\mcode{../include/lab/fibonacci.h}
\mcode{../source/fibonacci.cpp}
\newpage

\subsection{Метод парабол}
\mcode{../include/lab/parabola.h}
\mcode{../source/parabola.cpp}

\subsection{Комбинированный метод Брента}
\mcode{../include/lab/brent.h}
\mcode{../source/brent.cpp}

\end{document}

