\documentclass[12pt,a4paper]{article}
\usepackage[utf8]{inputenc}
\usepackage[T2A]{fontenc}
\usepackage[english,russian]{babel}
\usepackage[a4paper, mag=1000, left=1.5cm, right=2cm, top=2cm, bottom=2cm, headsep=0.7cm, footskip=1cm]{geometry}
\usepackage{amsmath}
\usepackage{pgfplots, colortbl}
\usepackage{makecell}
\usepackage{multicol}
\usepackage{pgfplotstable}
\pgfplotsset{compat=1.16}
\usepackage{minted}
\usepackage{listings}
\usepackage{lstfiracode}
\usepackage{caption}
\usepackage{mathrsfs}
\usepackage{placeins}
\usepackage{graphicx}

\usemintedstyle{colorful}
\newenvironment{code}{\captionsetup{type=listing}}{}


\newcommand{\itertable}[2]{
	\FloatBarrier
	\begin{table}[h]
		\centering
		\caption*{Количество точек}
		\pgfplotstabletypeset[
			every even row/.style=
				{before row={\rowcolor[gray]{0.95}}},
			string type,
			columns/point/.style={column name=Начальная точка, column type={|c|}},
			columns/f1/.style={column name=$f_1$, column type={c|}},
			columns/f2/.style={column name=$f_2$, column type={c|}},
			columns/f3/.style={column name=$f_3$, column type={c|}},
			columns/f4/.style={column name=$f_4$, column type={c|}},
			every head row/.style={before row=\hline, after row=\hline},
			every last row/.style={after row=\hline}
		]{#1}
	\end{table}
	\FloatBarrier
}

\newcommand{\dtable}[2]{
	\FloatBarrier
	\begin{table}[h]
		\centering
		\caption*{#2}
		\pgfplotstabletypeset[
			every even row/.style=
			{before row={\rowcolor[gray]{0.95}}},
			string type,
			columns/iter/.style={column name=$№$, column type={|c|}},
			columns/iter/.style={column name=$\alpha$, column type={c|}},
			every head row/.style={before row=\hline, after row=\hline},
			every last row/.style={after row=\hline}
		]{#1}
	\end{table}
	\FloatBarrier
}

\newcommand{\methodtable}[2]{
	\FloatBarrier
	\begin{table}[h]
		\centering
		\caption*{Количество точек}
		\pgfplotstabletypeset[
			every even row/.style=
			{before row={\rowcolor[gray]{0.95}}},
			string type,
			columns/method/.style={column name=Метод, column type={|c|}},
			columns/f1/.style={column name=$f_1$, column type={c|}},
			columns/f2/.style={column name=$f_2$, column type={c|}},
			every head row/.style={before row=\hline, after row=\hline},
			every last row/.style={after row=\hline}
		]{#1}
	\end{table}
	\FloatBarrier
}



\newcommand{\loggraph}[3]{
	\begin{center}
		\begin{tikzpicture}
			\begin{semilogxaxis}[
					title = {График зависимости количества итераций метода от размерности},
					xlabel = $\log n$,
					ylabel = \(\log iterations\),
					ylabel style={rotate=-90},
					ymode = log,
					legend pos=outer north east
				]
				\addplot table [x={n}, y={iter}, /pgf/number format/read comma as period] {#1};
				\addplot table [x={n}, y={iter}, /pgf/number format/read comma as period] {#2};
				\addplot table [x={n}, y={iter}, /pgf/number format/read comma as period] {#3};
				\addlegendentry{диагональное преобладание}
				\addlegendentry{обратный знак}
				\addlegendentry{матрицы Гильберта}
			\end{semilogxaxis}
		\end{tikzpicture}
	\end{center}
}


\newcommand{\img}[1] {
	\begin{center}
		\includegraphics[width=.7\linewidth, height=.4\textheight]{#1}
	\end{center}
}
\newcommand{\mcode}[2]{
	\begin{code}
		\caption*{#1}
		\inputminted[breaklines=true, xleftmargin=1em, linenos, frame=single, framesep=10pt, fontsize=\footnotesize]{cpp}{#2}
	\end{code}
	\newpage
}


\begin{document}



\begin{titlepage}
	\begin{center}
		\textsc{Национальный исследовательский университет ИТМО\\
			Прикладная математика и информатика}\\[5cm]

		\huge{Методы оптимизации\\[6mm]
			\large Отчет по лабораторной работе №4\\
			``Изучение алгоритмов метода Ньютона и его модификаций, в том числе
			квазиньютоновских методов''\\[4cm]

		}
	\end{center}

	\begin{flushright}
		\begin{minipage}{0.25\textwidth}
			Выполнили:\\[2mm]
			Михайлов Максим\\
			Загребина Мария\\
			Кулагин Ярослав\\[2mm]
			Команда:

			\(\forall \bar R \in \mathscr{R}^n : \mathrm{\textbf{R}}(\bar R) \in \mathscr{R}\)

			(КаМаЗ)\\[2mm]
			Группа: M3237
		\end{minipage}
	\end{flushright}

	\vfill
	\begin{center}
		Санкт-Петербург, \today
	\end{center}
\end{titlepage}





\section{Цель}
\begin{enumerate}
	\item Разработать программы для безусловной минимизации функций
	многих переменных
	\item Реализовать метод Ньютона
		\begin{itemize}
			\item классический
			\item с одномерным поиском
			\item с направлением спуска
		\end{itemize}
	\item Продемонстрировать работу методов на 2-3 функциях, исследовать влияние выбора начального приближения на результат
	\item Исследовать работу методов на двух функциях с заданным начальным приближением
		\begin{itemize}
			\item \(f(x) = x_1^2 + x_2^2 - 1.2x_1x_2,\ x^0 = (4, 1)^T\)
			\item \(f(x) = 100(x_2 - x_1^2)^2 + (1 - x_1)^2,\ x^0 = (-1.2, 1)^T\)	
		\end{itemize}
	\item Реализовать метод Давидона-Флетчера-Пауэлла и метод Пауэлла и сравнить с наилучшим методом Ньютона
\end{enumerate}

\section{Ход работы}

Обозначение цветов на иллюстрациях:\\
Классический метод - зеленый\\
Метод Ньютона с одномерным поиском - голубой\\
Метод Ньютона с направлением спуска - оранжевый\\
Во всех измерениях для одномерного поиска использовался метод Брента.

\subsection{Метод Ньютона}

\(\varepsilon = 10^{-5}\)\\
Начальная точка $(0.1, 0,1)$\\
\(f_1 = 108x^2 + 116y^2 + 80xy + 43x + 33y - 211\)\\
\img{img/f1.png}

\(f_2 = sin(x) + cos(y) + 0.3y^2+ 0.3x^2 + 0.1y\)
\img{img/sincos.png}

Классический метод Ньютона\\
\itertable{data/task1/classic.txt}

Метод Ньютона с одномерным поиском\\
\itertable{data/task1/search.txt}

\dstable

Для $f_1\ \alpha = 1$\\

Метод Ньютона с направлением спуска\\
\itertable{data/task1/descent.txt} 


Если начальное приближение недостаточно близко к решению, то метод Ньютона может не сойтись.\\
Выбор начального приближения влияет на количество итераций методов.\\
Так как матрица Гессе квадратичной функции положительно определена, все методы сходятся за одну итерацию и накладываются на графике.

\subsection{Исследование на заданных функциях}
\(f_1 = x_1^2 + x_2^2 - 1.2x_1x_2,\ x^0 = (4, 1)^T\) \\
\img{img/f2_1.png}

\(f_2 = 100(x_2 - x_1^2)^2 + (1 - x_1)^2,\ x^0 = (-1.2, 1)^T\)	
\img{img/f2_2.png}

\methodtable{data/task1_2/all.txt} 

По результатам измерений на данных функциях самый быстрый метод Ньютона - классический, но он не гарантирует сходимость, поэтому в следующем задании с квазиньютоновскими методами будет сравниваться метод с направлением спуска.\\
 Все методы работают гораздо медленнее на овражной функции $f_2$. По сравнению с наискорейшим спуском из 2-ой лабораторной работы, методы используют меньшее число итераций, и не так сильно зависят от числа обусловленности.

\subsection{Квазиньютоновские методы}
Начальная точка $(-1.2, 1)$
\(f_1 = 100(x_2 - x_1^2)^2 + (1 - x_1)^2,\)\\
\img{img/f2_2_1.png}

\(f_2 = (x_1^2 + x_2 - 11)^2 + (x_1 + x_2^2 - 7)^2)\)\\
\img{img/f2_2_2.png}

\(f_3 = (x_1 + 10x_2)^2 + 5(x_3 - x_4)^2 + (x_2 - 2x_3)^4 + 10(x_1 - x_4)^4\)\\

\(f_4 = 100 - \frac{2}{1 + (\frac{x_1 - 1}{2})^2 + (\frac{x_2 - 1}{3})^2} - \frac{1}{1 + (\frac{x_1 - 2}{2})^2 + (\frac{x_2 - 1}{3})^2}\)\\
\img{img/f2_2_4.png}

Метод Ньютона с направлением спуска\\
\itertable{data/task3/classic.txt}

Метод Давидона-Флетчера-Пауэлла\\
\itertable{data/task3/dfp.txt}

Метод Пауэлла\\
\itertable{data/task3/powell.txt}

Все методы работают в несколько раз медленнее на функции с большим числом обусловленности и многомерной. Траектории всех методов оказались одинаковыми.\\
Метод Ньютона с направлением спуска сходится за меньшее количество итераций, чем квазиньютоновские методы, т.к. с большей точностью выбирает направление для движения.

\newpage 

\section{Выводы}

\begin{enumerate}
	\item Классический метод Ньютона сходится не для каждого начального приближения, так как в нем нет оптимизации по выбору $\alpha$. Остальные методы Ньютона более надежные и показали похожий результат.
	
	\item Выбор начального приближения, как и число обусловленности, влият на количество итераций методов.
	
	\item Методы ДФП и Пауэлла работают примерно с одинаковой скоростью и находят значения с заданной точностью на всех рассмотренных функциях.
	
	\item Метод Ньютона с направлением спуска находит минимум быстрее остальных методов, расмотренных в данной работе.
\end{enumerate}

\newpage

\section{Исходный код}

\mcode{Функция}{../../include/lab4/n_function_impl.h}
\mcode{Функция}{../../source/lab4/n_function_impl.cpp}

\mcode{Классический метод Ньютона}{../../include/lab4/classic_newton.h}
\mcode{Классический метод Ньютона}{../../source/lab4/classic_newton.cpp}

\mcode{метод Давидона-Флетчера-Пауэлла}{../../include/lab4/dfp.h}
\mcode{метод Давидона-Флетчера-Пауэлла}{../../source/lab4/dfp.cpp}

%\mcode{Метод Ньютона с одномерным поиском}{../../include/lab4/1d_search_newton.h}
%\mcode{Метод Ньютона с одномерным поиском}{../../source/lab4/1d_search_newton.cpp}

\mcode{Метод Ньютона с направлением спуска}{../../include/lab4/dfp.h}
\mcode{Метод Ньютона с направлением спуска}{../../source/lab4/dfp.cpp}

\mcode{Метод Пауэлла}{../../include/lab4/powell.h}
\mcode{Метод Пауэлла}{../../source/lab4/powell.cpp}
\end{document}
