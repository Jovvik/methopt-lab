\documentclass[12pt,a4paper]{article}
\usepackage[utf8]{inputenc}
\usepackage[T2A]{fontenc}
\usepackage[english,russian]{babel}
\usepackage[a4paper, mag=1000, left=1.5cm, right=2cm, top=2cm, bottom=2cm, headsep=0.7cm, footskip=1cm]{geometry}
\usepackage{amsmath}
\usepackage{pgfplots, colortbl}
\usepackage{makecell}
\usepackage{multicol}
\usepackage{pgfplotstable}
\pgfplotsset{compat=1.16}
\usepackage{minted}
\usepackage{listings}
\usepackage{lstfiracode}
\usepackage{caption}
\usepackage{mathrsfs}
\usepackage{placeins}
\usepackage{graphicx}

\usemintedstyle{colorful}
\newenvironment{code}{\captionsetup{type=listing}}{}


\newcommand{\itertable}[2]{
	\FloatBarrier
	\begin{table}[h]
		\centering
		\caption*{Количество итераций}
		\pgfplotstabletypeset[
			every even row/.style=
				{before row={\rowcolor[gray]{0.95}}},
			string type,
			columns/point/.style={column name=Начальная точка, column type={|c|}},
			columns/f1/.style={column name=$f_1$, column type={c|}},
			columns/f2/.style={column name=$f_2$, column type={c|}},
			columns/f3/.style={column name=$f_3$, column type={c|}},
			columns/f4/.style={column name=$f_4$, column type={c|}},
			every head row/.style={before row=\hline, after row=\hline},
			every last row/.style={after row=\hline}
		]{#1}
	\end{table}
	\FloatBarrier
}

\newcommand{\dtable}[2]{
	\FloatBarrier
	\begin{table}[h]
		\centering
		\caption*{#2}
		\pgfplotstabletypeset[
			every even row/.style=
			{before row={\rowcolor[gray]{0.95}}},
			string type,
			columns/iter/.style={column name=$№$, column type={|c|}},
			columns/iter/.style={column name=$\alpha$, column type={c|}},
			every head row/.style={before row=\hline, after row=\hline},
			every last row/.style={after row=\hline}
		]{#1}
	\end{table}
	\FloatBarrier
}

\newcommand{\methodtable}[2]{
	\FloatBarrier
	\begin{table}[h]
		\centering
		\caption*{Количество итераций}
		\pgfplotstabletypeset[
			every even row/.style=
			{before row={\rowcolor[gray]{0.95}}},
			string type,
			columns/method/.style={column name=Метод, column type={|c|}},
			columns/f1/.style={column name=$f_1$, column type={c|}},
			columns/f2/.style={column name=$f_2$, column type={c|}},
			every head row/.style={before row=\hline, after row=\hline},
			every last row/.style={after row=\hline}
		]{#1}
	\end{table}
	\FloatBarrier
}


\newcommand{\fullgraph}[4]{
	\begin{center}
		\begin{tikzpicture}
			\begin{axis}[
					title = {График зависимости логарифма относительной погрешности от $k$},
					xlabel = $k$,
					ylabel = \(\log \delta x\),
					legend pos=outer north east,
					ylabel style={rotate=-90},
					ymode = log
				]
				\addplot table [x={k}, y={rel}, /pgf/number format/read comma as period] {data/task2/n#1.csv};
				\addplot table [x={k}, y={rel}, /pgf/number format/read comma as period] {data/task2/n#2.csv};
				\addplot table [x={k}, y={rel}, /pgf/number format/read comma as period] {data/task2/n#3.csv};
				\addplot table [x={k}, y={rel}, /pgf/number format/read comma as period] {data/task2/n#4.csv};
				\addlegendentry{$n = #1$}
				\addlegendentry{$n = #2$}
				\addlegendentry{$n = #3$}
				\addlegendentry{$n = #4$}
			\end{axis}
		\end{tikzpicture}
		\begin{tikzpicture}
			\begin{axis}[
					title = {График зависимости логарифма относительной погрешности от $cond$},
					xlabel = $\log cond$,
					ylabel = \(\log \delta x\),
					legend pos=outer north east,
					ylabel style={rotate=-90},
					xmode = log,
					ymode = log,
				]
				\addplot table [x={cond}, y={rel}, /pgf/number format/read comma as period] {data/task2/n#1.csv};
				\addplot table [x={cond}, y={rel}, /pgf/number format/read comma as period] {data/task2/n#2.csv};
				\addplot table [x={cond}, y={rel}, /pgf/number format/read comma as period] {data/task2/n#3.csv};
				\addplot table [x={cond}, y={rel}, /pgf/number format/read comma as period] {data/task2/n#4.csv};
				\addlegendentry{$n = #1$}
				\addlegendentry{$n = #2$}
				\addlegendentry{$n = #3$}
				\addlegendentry{$n = #4$}
			\end{axis}
		\end{tikzpicture}
	\end{center}
}

\newcommand{\graph}[1]{
	\begin{center}
		\begin{tikzpicture}
			\begin{axis}[
					title = {График зависимости логарифма относительной погрешности от размерности матрицы},
					xlabel = $n$,
					ylabel = \(\log \delta x\),
					ylabel style={rotate=-90},
					ymode = log
				]
				\addplot table [x={n}, y={rel}, /pgf/number format/read comma as period] {#1};
			\end{axis}
		\end{tikzpicture}
	\end{center}
}

\newcommand{\loggraph}[3]{
	\begin{center}
		\begin{tikzpicture}
			\begin{semilogxaxis}[
					title = {График зависимости количества итераций метода от размерности},
					xlabel = $\log n$,
					ylabel = \(\log iterations\),
					ylabel style={rotate=-90},
					ymode = log,
					legend pos=outer north east
				]
				\addplot table [x={n}, y={iter}, /pgf/number format/read comma as period] {#1};
				\addplot table [x={n}, y={iter}, /pgf/number format/read comma as period] {#2};
				\addplot table [x={n}, y={iter}, /pgf/number format/read comma as period] {#3};
				\addlegendentry{диагональное преобладание}
				\addlegendentry{обратный знак}
				\addlegendentry{матрицы Гильберта}
			\end{semilogxaxis}
		\end{tikzpicture}
	\end{center}
}

\newcommand{\densegraph}[2]{
	\begin{center}
		\begin{tikzpicture}
			\begin{axis}[
					title = {График зависимости логарифма относительной погрешности от размерности матрицы},
					xlabel = $n$,
					ylabel = \(\log \delta x\),
					legend pos=outer north east,
					ylabel style={rotate=-90}
				]
				\addplot table [x={n}, y={rel}, /pgf/number format/read comma as period] {#1};
				\addplot table [x={n}, y={rel}, /pgf/number format/read comma as period] {#2};
				\addlegendentry{метод Гаусса для плотных матриц}
				\addlegendentry{метод Гаусса для LU-разложения}
			\end{axis}
		\end{tikzpicture}
	\end{center}
}


\newcommand{\img}[1] {
	\begin{center}
		\includegraphics[width=.7\linewidth, height=.4\textheight]{#1}
	\end{center}
}
\newcommand{\mcode}[2]{
	\begin{code}
		\caption*{#1}
		\inputminted[breaklines=true, xleftmargin=1em, linenos, frame=single, framesep=10pt, fontsize=\footnotesize]{java}{path}
	\end{code}
	\newpage
}
