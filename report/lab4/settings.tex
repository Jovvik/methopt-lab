\documentclass[12pt,a4paper]{article}
\usepackage[utf8]{inputenc}
\usepackage[T2A]{fontenc}
\usepackage[english,russian]{babel}
\usepackage[a4paper, mag=1000, left=1.5cm, right=2cm, top=2cm, bottom=2cm, headsep=0.7cm, footskip=1cm]{geometry}
\usepackage{amsmath}
\usepackage{pgfplots, colortbl}
\usepackage{makecell}
\usepackage{multicol}
\usepackage{pgfplotstable}
\pgfplotsset{compat=1.16}
\usepackage{minted}
\usepackage{listings}
\usepackage{lstfiracode}
\usepackage{caption}
\usepackage{mathrsfs}
\usepackage{placeins}
\usepackage{graphicx}
\usepackage{svg}
\usepackage{caption}
\usepackage{subcaption}

\setlength{\parindent}{0pt}

\usemintedstyle{colorful}
\newenvironment{code}{\captionsetup{type=listing}}{}


\newcommand{\itertable}[2]{
	% \FloatBarrier
	\begin{table}[H]
		\centering
		\caption*{#2}
		\pgfplotstabletypeset[
			every even row/.style=
				{before row={\rowcolor[gray]{0.95}}},
			string type,
			columns/point/.style={column name=Начальная точка, column type={|c|}},
			columns/f1/.style={column name=$f_1$, column type={c|}},
			columns/f2/.style={column name=$f_2$, column type={c|}},
			columns/f3/.style={column name=$f_3$, column type={c|}},
			columns/f4/.style={column name=$f_4$, column type={c|}},
			every head row/.style={before row=\hline, after row=\hline},
			every last row/.style={after row=\hline}
		]{#1}
	\end{table}
	% \FloatBarrier
}

\newcommand{\dstable}{
	\FloatBarrier
	\begin{table}[h]
		\centering
		\caption*{$\alpha$, полученные одномерным поиском для $f_2$}
		\pgfplotstabletypeset[
			every even row/.style=
				{before row={\rowcolor[gray]{0.95}}},
			string type,
			columns/iter/.style={column name=№, column type={|c|}},
			columns/p1/.style={column name=(0.1; 0.1), column type={c|}},
			columns/p2/.style={column name=(1; 1), column type={c|}},
			columns/p3/.style={column name=(2; 2), column type={c|}},
			columns/p4/.style={column name=(-5; -5), column type={c|}},
			columns/p5/.style={column name=(10; 10), column type={c|}},
			every head row/.style={before row=\hline, after row=\hline},
			every last row/.style={after row=\hline}
		]{data/task1/alpha.txt}
	\end{table}
	\FloatBarrier
}

\newcommand{\twotables}[3]{
	\begin{minipage}{.3\textwidth}
		\centering
		\dstable{data/task1/alpha/p#2.txt}
	\end{minipage}
}
\newcommand{\methodtable}[2]{
	\FloatBarrier
	\begin{table}[h]
		\centering
		% \caption*{}
		\pgfplotstabletypeset[
			every even row/.style=
				{before row={\rowcolor[gray]{0.95}}},
			string type,
			columns/method/.style={column name=Метод, column type={|c|}},
			columns/f1/.style={column name=$f_1$, column type={c|}},
			columns/f2/.style={column name=$f_2$, column type={c|}},
			every head row/.style={before row=\hline, after row=\hline},
			every last row/.style={after row=\hline}
		]{#1}
	\end{table}
	\FloatBarrier
}



\newcommand{\loggraph}[3]{
	\begin{center}
		\begin{tikzpicture}
			\begin{semilogxaxis}[
					title = {График зависимости количества итераций метода от размерности},
					xlabel = $\log n$,
					ylabel = \(\log iterations\),
					ylabel style={rotate=-90},
					ymode = log,
					legend pos=outer north east
				]
				\addplot table [x={n}, y={iter}, /pgf/number format/read comma as period] {#1};
				\addplot table [x={n}, y={iter}, /pgf/number format/read comma as period] {#2};
				\addplot table [x={n}, y={iter}, /pgf/number format/read comma as period] {#3};
				\addlegendentry{диагональное преобладание}
				\addlegendentry{обратный знак}
				\addlegendentry{матрицы Гильберта}
			\end{semilogxaxis}
		\end{tikzpicture}
	\end{center}
}


\newcommand{\img}[1] {
	\begin{center}
		\includegraphics[width=.7\linewidth, height=.4\textheight]{#1}
	\end{center}
}
\newcommand{\mcode}[2]{
	\begin{code}
		\caption*{#1}
		\inputminted[breaklines=true, xleftmargin=1em, linenos, frame=single, framesep=10pt, fontsize=\footnotesize]{cpp}{#2}
	\end{code}
	\newpage
}
